% Options for packages loaded elsewhere
\PassOptionsToPackage{unicode}{hyperref}
\PassOptionsToPackage{hyphens}{url}
%
\documentclass[
]{article}
\usepackage{amsmath,amssymb}
\usepackage{iftex}
\ifPDFTeX
  \usepackage[T1]{fontenc}
  \usepackage[utf8]{inputenc}
  \usepackage{textcomp} % provide euro and other symbols
\else % if luatex or xetex
  \usepackage{unicode-math} % this also loads fontspec
  \defaultfontfeatures{Scale=MatchLowercase}
  \defaultfontfeatures[\rmfamily]{Ligatures=TeX,Scale=1}
\fi
\usepackage{lmodern}
\ifPDFTeX\else
  % xetex/luatex font selection
\fi
% Use upquote if available, for straight quotes in verbatim environments
\IfFileExists{upquote.sty}{\usepackage{upquote}}{}
\IfFileExists{microtype.sty}{% use microtype if available
  \usepackage[]{microtype}
  \UseMicrotypeSet[protrusion]{basicmath} % disable protrusion for tt fonts
}{}
\makeatletter
\@ifundefined{KOMAClassName}{% if non-KOMA class
  \IfFileExists{parskip.sty}{%
    \usepackage{parskip}
  }{% else
    \setlength{\parindent}{0pt}
    \setlength{\parskip}{6pt plus 2pt minus 1pt}}
}{% if KOMA class
  \KOMAoptions{parskip=half}}
\makeatother
\usepackage{xcolor}
\usepackage[margin=1in]{geometry}
\usepackage{graphicx}
\makeatletter
\def\maxwidth{\ifdim\Gin@nat@width>\linewidth\linewidth\else\Gin@nat@width\fi}
\def\maxheight{\ifdim\Gin@nat@height>\textheight\textheight\else\Gin@nat@height\fi}
\makeatother
% Scale images if necessary, so that they will not overflow the page
% margins by default, and it is still possible to overwrite the defaults
% using explicit options in \includegraphics[width, height, ...]{}
\setkeys{Gin}{width=\maxwidth,height=\maxheight,keepaspectratio}
% Set default figure placement to htbp
\makeatletter
\def\fps@figure{htbp}
\makeatother
\setlength{\emergencystretch}{3em} % prevent overfull lines
\providecommand{\tightlist}{%
  \setlength{\itemsep}{0pt}\setlength{\parskip}{0pt}}
\setcounter{secnumdepth}{-\maxdimen} % remove section numbering
\usepackage{setspace}\doublespacing
\ifLuaTeX
  \usepackage{selnolig}  % disable illegal ligatures
\fi
\IfFileExists{bookmark.sty}{\usepackage{bookmark}}{\usepackage{hyperref}}
\IfFileExists{xurl.sty}{\usepackage{xurl}}{} % add URL line breaks if available
\urlstyle{same}
\hypersetup{
  pdftitle={Is This House Price Index any Good?},
  pdfauthor={Andy Krause and Reid Johnson},
  hidelinks,
  pdfcreator={LaTeX via pandoc}}

\title{Is This House Price Index any Good?}
\author{Andy Krause and Reid Johnson}
\date{2023-12-10}

\begin{document}
\maketitle

\hypertarget{abstract}{%
\subsubsection{Abstract}\label{abstract}}

House price indices are a widely used tool for understanding the
aggregate movements in housing markets. As such they play an important
role in decision-making by government bodies, federal and local, as well
as individual market actors. There is robust collection of prior
research on methods for index creation; however, a much smaller set of
work addressing fundamental questions of index quality. In this work, we
present a broad ranging set of criteria and metrics to judge the fitness
of an index. Our empirical examples compare two common approach --
Repeat Sales and Hedonic -- against a novel Interpretable Machine
Learning (IML) method. We find that index preference is a question of
use case of the index combined with measurable objective qualities of
the index itself.

\hypertarget{introduction}{%
\subsection{Introduction}\label{introduction}}

\emph{Traditionally, house price indexes have been derived from highly
interpretable (statistical) modeling approaches such as regression
models. Both the repeat sales and the hedonic approach -- the two most
commonly published approaches (Hill 2012; MacGuire et al 2013) -- are
regression-based models. Standard statistical models are a good fit for
this task as the coefficient estimates are easily convertible into
standardized price indexes. House price index generation is viewed as an
inferential endeavor in which the attribution of the effects of time on
market movements is sought, rather than a pure prediction problem. As a
result, many of the rapidly growing set of machine learning algorithms
-- e.g.~support vector machines, random forests and neural networks --
have not been used in the production of price indexes due to the fact
that they do not directly and/or easily attribute price impacts to the
variables or features in the model. However, with the rise of
interpretability methods (Ribiero et al.~2016; Doshi-Velez and Kim 2017;
Molnar 2019), these `black-box' models can be made explainable and
suitable for a more diverse set of tasks. }

\emph{This paper highlights the use of partial dependence -- a
model-agnostic interpretability method (Molnar 2019) -- to generate
house price indexes from a machine learning and inherently
non-interpretable model. One of the major appeals of using a
model-agnostic approach is that any underlying model class could be used
on the data. In this work I use a random forest, one of the more common
and intuitive machine learning models. However, this choice of model
class is only for convenience, as a neural network, for example, could
just as easily have been used. Along with an explanation of the method
and examples, the results from the application of a model-agnostic
interpretability method are compared to the more traditional repeat
sales and hedonic model approaches. The findings suggest that the
interpretable random forest (IRF) appraoch to house price generation is
competitive with (and occasionally prefereable to) the standard
approaches across measure of accuracy, volatility and revision on a set
of data from the city of Seattle. }

\emph{The remainder of this work is organized as follows: Section two
provides a brief literature review focused on applying machine learning
approaches to the task of house price indexing. I then discuss the
interpretable random forest (IRF) approach to creating a house price
index and provide details using a dataset from the \texttt{hpiR} R
package: Seattle, WA homes sales in the 2010 through 2016 period. In
section four, the interpretable random forest method is compared to more
traditional models across three metrics -- accuracy, volatility and
revision. Finally, I conclude with a discussion of the results and the
reproducility of this work. }

\hypertarget{previous-work}{%
\subsection{Previous Work}\label{previous-work}}

Since the seminal Bailey et al.~(1963) study there has been considerable
and sustained research effort put into comparing and improving competing
methods for generating house price indexes. Published work in this
sub-field of housing economics is generally focused on one or more of
four aims: 1) Comparison of model differences (Case et al 1991; Crone \&
Voith 1992; Meese and Wallace 1997; Nagaraja et al 2014; Bourassa et al
2016); 2) Identification and correction of estimation issues or problems
(Abraham \& Schauman 1991; Haurin \& Henderschott 1991; Clapp et al
1992; Case et al 1997; Steele \& Goy 1997; Gatzlaff \& Haurin 1997,
1998; Munneke \& Slade 2000); 3) Creation of local or submarket indexes
(Goodman 1978; Hill et al 1997; Gunterman et al 2016; Bogin et al 2019;
Ahlfeldt et al 2023); and/or 4) Development of a new model or estimator
(Case \& Quigley 1991; Quigley 1995; Hill et al 1997; Englund et
al.~1998, McMillen 2012; Bokhari \& Geltner 2012; Bourassa et al 2016;
Xu and Zhang 2022).

Measuring the quality of house price indexes has received considerable
less attention in the research. Part of this likely stems from the fact
that a house price index is created to measure a phenomena without an
observable ground truth -- the price or value movements of a given
market -- and therefore is, under any method or approach, a proxy at
best. This inherent fact also means that any approach to measuring model
quality comes with shortcomings and/or assumptions.

In the existing work, three general criteria have been applied to
evaluating house price indexes; one based on input and method and two on
outputs. The input-based criteria asks the general question of ease of
construction. The question of the constructability of house price
indexes can be divided into two sub-questions or criteria. First, how
easy or available is the necessary data for constructing the index? In
reality, this reduces to a question of the availability of granular
sales and home attribute data.

Under this framework, the easiest index to create is one that relies
only on aggregated prices, such as mean or median of observed prices in
a given market over a given time period. This results in simplistic mean
or median value indexes; approaches that are often used as baselines in
comparative work (cite). Next are indexes based on repeat transaction of
the same home. Repeat sales are appealing as they solve some of the
constant quality issues that fully aggregated indexes create and they
only require granular sales data -- the sale price, the sale data and a
unique home identifier such as address. Finally, we have a wide variety
of approach, broadly characterized in the literature as hedonic methods
(Hill 2013). These approaches require both transaction and home
attribute data for all transactions, the heaviest data collection burden
of the three.

The second question to ask under the `constructability' criteria is the
model estimator itself. Again, a fully aggregated approach like a median
or mean offers the lowest barrier. Traditionally, repeat sales models
(cite) utilized basic linear models (OLS) which are straight-forward to
both create and understand. More recent work (Bourassa\ldots cite) has
developed more advanced or robust approaches to repeat sale estimation
but they are still rooted in linear regressor. The broad class of
`hedonic' approach again offer the most comlexity in estimation. Within
this class, constructability can vary widely from still quite
straightforward in the case of linear hedonic model with dummy
estimators (ex. Hill and Trojanek 2022) to very complex method derives
from neural network estimators (ex. Xu and Zhang 2023). Hedonic
estimators offer complexity in the form of the choice of model
specification (independent variables), structure and, in the case of
machine learning-based approaches, the selection of hyperparameters.
Together, all of the these decision about the model combined with the
potential for greatly increased compute times as the methodology becomes
more specifialized means that hedonic approach present the highest
barriers to constructability.

In sum, the constructability criteria is a combination of data
requirements and estimator complexity, interpretability and computation
time.

Output metrics

The most common output metric in the literature is that of index
accuracy. As noted above, there is no exact ground truth of the movement
of aggregate house prices in a region so any approach here is a proxy,
with limitations. The favored approach in the existing literature is to
test the ability of an index to predict the second sale of repeat
transaction pair. This approach takes the first sale of a pair, indexes
that price forward by the given index to the date of the repeat
transaction and then computes an error measure where the error is the
difference in the predicted value and the actual price of the second
sale in the pair. This is an appealing approach as individual error
metrics at the home level allow for computation of standard metrics like
root mean squared error (RMSE), mean absolute error (MAE) and others
that are common in the evaluation of Automated Valuation Models, other
hedonic pricing models and many/most regression-based machine learning
tasks(cite that paper on ways to measure AVMs).

A downfall of this particular approach to accuracy; however, is that it
relies solely on repeat transactions. As noted by others (Hill 2013;
\ldots), repeat transactions are not a random sample of all homes in the
market nor are they are random sample even of those homes that sell.
Additionally, repeat transactions themselves can be subject to
violations of constant quality in the case of renovation and/or
depreciation (Steele and Gray 1997).

A second approach to measure the objective qualities of an index looks
at the revision of an index over time (\ldots, Van de Minne et al 2020).
Revisions occurs when the re-estimation of the index over time results
in changes to prior estimates. For example, take an index with a value
of 134.2 in August of 2023. If when the index is re-estimated in
September and that value changes to 134.4, that is a revision of 0.2
points. Revision, particularly those that are substantial can create
harmful impacts on users of indices in applied and/or policy
persectives. It is generally accepted that index revision should be
minimized (cite)

In summary, the existing work on measuring index quality uses three
criteria to evaluation house price indexes: 1) a measure of the ease of
constructability; 2) a measure of accuracy in the form of predicting the
second of a repeat transaction; and 3) a measure of the extent to which
the index values revise as a series of indexes are created over time. We
use this existing framwork as a basis for our extensions discussed in
the section below.

\hypertarget{extending-the-measure-of-index-evaluation}{%
\subsubsection{Extending The Measure of Index
Evaluation}\label{extending-the-measure-of-index-evaluation}}

One perspective missing from the current framework for index evaluation
is that of the end use or user. The purpose to which the index is being
put should be considered. To bring this perspective we add three
addition evaluation criteria to the existing three: 1) Relative
predictive accuracy improvement; 2) Volatility; 3) Local specificity.

As noted above, the use of repeat transactions as the benchmark for
predictive accuracy suffers from many of the same limitations as repeat
transaction models -- namely sample selection and constant quality
issues stemming from renovation and/or depreciation. This approach also
ignores one of the growing uses to which house price indexes are
employed, that of supporting automated and/or manual home pricing
exercises. Some approaches to the construction of automated valuation
models (AVMs) will first time-adjust all sales in the training data (the
target values) to the price as if the home sold at the time of the AVM
training. By doing so, a model can focus on distinguishing the
cross-section variation between homes while relying on an outside model
to make temporal adjustments.

From the manual side, indexes are often used to make time adjustments in
traditional adjustment grids or comparable market analyses (CMAs). These
are use cases where a small sample of recent, similar sales are used to
create a value estimate. Those recent sale prices are generally adjusted
forward in time to the date of valuation using an index.

Considering these growing use cases of home price indexes, we offer an
alternative to the commonly used repeat transaction accuracy methods. In
this alternative, we compute the relative accuracy improvement of an
index over a baseline model that has no temporal adjustments. This is a
two-step process in which first an baseline hedonic pricing model is
estimated in which time is completely ignored; there are no temporal
variables. We measure the predictive accuracy of this model as the
baseline. We then time-adjusted all sale prices in the data to the
valuation date of the model and re-estimate it, again measuring the
predictive ability. The relative change in accuracy between the baseline
model (no time controls) and the model with index-adjusted sale prices
represents a comparative measure of the index's ability to improve a
valuation model. This offers another measure of the index's accuracy
where accuracy is a proxy for its ability to track the actual
(unobserved) movements in the market.

To the extent that home price indexes are meant to represent an estimate
of the aggregate movement of all home values in a region or market, a
desirable characteristic is that the index is not overly noisy. Noisy
here mean that that index `chases' or is overly impacted by one or a
single data points that are likely not representative of the underlying
global phenomenon. In the machine learning domain this is often
represented by a model that has great fit on in-sample data but has a
large reduction in accuracy when applied to new, out of sample data
(cite). Likewise, with an index, a high noise to trend ratio could be
indicative of the same problem.

In addition to the fundamental, estimator-based issues stemming from
volatility, a highly noisy index is also hard to rationalize to users of
indexes, particularly those looking to make important financial
decisions based on the (perceived) market changes suggested by the
index. Simply put, it is hard to build trust with users when indexes
have a high noise-to-signal ratio.

Conversely, we shouldn't look to fully minimize volatility either as,
taken to an extreme, a perfectly flat index would optimize volatility
reduction but at the cost of being useful at all. It is a delicate
balance. In this work, we measure volatilty by -------------, but look
at the comparison only relatively. We balance off volatility against
other measure in a subjective, criteria-ranking approach not as an
objective measure to be optimized. As a result, it operates somewhat
like the `constructability' criteria -- something that should be taken
into account by developers and users, but not a criteria to necessarily
be optimized against.

Finally, prior work has established the need to derive indexes for small
geographic regions or other local definitions of submarkets (cite). A
third measure of index quality is the relative change in the above
metrics when the index is applied to smaller subregions within the
larger market. We term this Local Specificity.

In the evaluations that follow we will focus on four objective measures
of house price index performance: 1) Absolute accuracy via repeat
transactions; 2) Relative accuracy via AVM improvement; 3) Revision; and
4) Volatility. With then look at all for of these measures at global and
local (to our data) levels in order to measure the Local Specificity or
ability of the index to discern local trends while also maintain the
four objective quality measures. In the discussion sections we'll also
layer in considerations of the `constructability' dimension on indexes
to highlight some of the tradeoffs

\hypertarget{reproducibility-and-software}{%
\subsubsection{Reproducibility and
Software}\label{reproducibility-and-software}}

This work is completely reproducible. All raw data, code and general
instructions to exactly recreate the analyses above is found at
\href{www.github.com/anonymousreauthor/irf_house_price_index}{https://www.github.com/anonymousreauthor/irf\_house\_price\_index}\footnote{NOTE
  to reviewers: This will be switched to my actual Github Repository
  after blind peer review.} All code is written in the R statistical
language. In addition to the \texttt{hpiR} package, which includes the
custom functions for the IRF models and the wrapper functions that make
for easy computation of accuracy, volatility and revision figures this
work also directly uses the following R packages: \texttt{caret}(Kuhn
2019), \texttt{dplyr}(Wickham et al 2019), \texttt{forecast}(Hyndman et
al 2019), \texttt{ggplot}(Wickham 2016), \texttt{imputeTS}(Moritz and
Bartz-Beielstein 2017), \texttt{knitr}(Xie 2019),
\texttt{lubridate}(Grolemund and Wickham 2011), \texttt{pdp}(Greenwell
2017), \texttt{purrr}(Henry and Wickham 2019), \texttt{ranger}(Wright
and Ziegler 2017), \texttt{robustbase}(Maechler et al 2019),
\texttt{tidyr}(Wickham and Henry 2019) and \texttt{zoo}(Zeileis and
Grothendieck 2005).

~ ~

\pagebreak

\hypertarget{references}{%
\subsection{References}\label{references}}

Abraham, J. M., \& Schauman, W. S. (1991). New evidence on home prices
from Freddie Mac repeat sales. \emph{Real Estate Economics}, 19(3),
333-352.

Ahlfeldt, G. M., Heblich, S., and Seidel, T. (2023) Micro-geographic
property price and rent indices, \emph{Regional Science and Urban
Economics}, 98.

Apley, D. (2016). Visualizing the Effects of Predictor Variables in
Black Box Supervised Learning Models.
\href{arxiv.org/abs/1612.08468}{https://arxiv.org/abs/1612.08468}

Bailey, M., Muth, R., \& Nourse, H. (1963). A Regression Method for Real
Estate Price Index Construction. \emph{Journal of the American
Statistical Association}, 58, 933-942.

Bogin, A. N., Doerner, W. M., Larson, W. D., \& others. (2016). Local
House Price Dynamics: New Indices and Stylized Facts \emph{FHFA Working
Paper}.

Bokhari, S. \& Geltner, D. J. (2012). Estimating Real Estate Price
Movements for High Frequency Tradable Indexes in a Scarce Data
Environment. \emph{The Journal of Real Estate Finance and Economics}
45(2), 533-543.

Bourassa, S., Cantoni, E., \& Hoesli, M. (2016). Robust hedonic price
indexes. \emph{International Journal of Housing Markets and Analysis},
9(1), 47-65.

Breiman, L. (2001) Random Forests. \emph{Machine Learning} 45(1), 5-32.
\href{https://link.springer.com/article/10.1023/A:1010933404324}{doi:10.1023/A:1010933404324}

Case, B., Pollakowski, H. O., \& Wachter, S. M. (1991). On choosing
among house price index methodologies. \emph{Real Estate Economics},
19(3), 286-307.

Case, B., Pollakowski, H. O., \& Wachter, S. (1997). Frequency of
transaction and house price modeling. \emph{The Journal of Real Estate
Finance and Economics}, 14(1), 173-187.

Case, B. \& Quigley, J. M. (1991). The dynamics of real estate prices.
\emph{The Review of Economics and Statistics}, 50-58.

Case, K. \& Shiller, R. (1987). Prices of Single Family Homes Since
1970: New Indexes for Four Cities. \emph{New England Economic Review},
Sept/Oct, 45-56.

Case, K. \& Shiller, R. (1989). The Efficiency of the Market for Single
Family Homes. \emph{The American Economic Review}, 79(1), 125-137.

Clapham, E., Englund, P., Quigley, J. M., \& Redfearn, C. L. (2006).
Revisiting the past and settling the score: index revision for house
price derivatives. \emph{Real Estate Economics}, 34(2), 275-302.

Clapp, J. M., \& Giaccotto, C. (1999). Revisions in Repeat-Sales Price
Indexes: Here Today, Gone Tomorrow? \emph{Real Estate Economics}, 27(1),
79-104.

Clapp, J. M., Giaccotto, C., \& Tirtiroglu, D. (1992). Repeat sales
methodology for price trend estimation: an evaluation of sample
selectivity. \emph{Journal of Real Estate Finance and Economics}, 5(4),
357-374.

Cohen, SB, Dror, G \& Ruppin, E (2005) Feature Selection Based on the
Shapley Value. in \emph{Proceedings of IJCAI}. pp.~1-6.

Crone, T. M., \& Voith, R. (1992). Estimating house price appreciation:
a comparison of methods. \emph{Journal of Housing Economics}, 2(4),
324-338.

Doshi-Velez, F. and Kim, B. (2017) Toward a Rigorous Science of
Interpretable Machine Learning. \emph{arXiv::1702.08608}.
\url{https://arxiv.org/abs/1702.08608}

Englund, P., Quigley, J. M., \& Redfearn, C. L. (1999). The choice of
methodology for computing housing price indexes: comparisons of temporal
aggregation and sample definition. \emph{The Journal of Real Estate
Finance and Economics}, 19(2), 91-112.

Eurostat (2013) Handbook on Residential Property Prices Indices (RPPIs).
\emph{Eurostat: Methodologies and Working Papers}
\href{https://op.europa.eu/en/publication-detail/-/publication/cee09dbc-bf48-4126-a7d1-0bb9c028f648/language-en}{doi:10.2785/34007}

Friedman, J. (2001). Greedy function approximation: A gradient boosting
machine. \emph{Annals of statistics} 1189-1232.

Gatzlaff, D. H., \& Haurin, D. R. (1997). Sample Selection Bias and
Repeat-Sales Index Estimates. \emph{The Journal of Real Estate Finance
and Economics}, 14, 33-50.

Gatzlaff, D. H., \& Haurin, D. R. (1998). Sample Selection and Biases in
Local House Value Indices. \emph{Journal of Urban Economics}, 43,
199-222.

Goldstein, A., Kapelner, A., Bleich, J. and Pitkin, E. (2014) Peeking
Inside the Black Box: Visualizing Statistical Learning with Plots of
Individual Conditional Expectiation.
\href{arxiv.org/pdf/1309.6392.pdf}{https://arxiv.org/pdf/1309.6392.pdf}

Goodman, A. C. (1978). Hedonic prices, price indices and housing
markets. \emph{Journal of Urban Economics}, 5, 471-484.

Greenwell, B. (2017). pdp: An R Package for Constructing Partial
Dependence Plots. \emph{The R Journal}, 9(1), 421--436. URL
\url{https://journal.r-project.org/archive/2017/RJ-2017-016/index.html}.

Gregorutti, B., Bertrand, M., and Saint-Pierre, P. (2017) Correlation
and variable importance in random forests. \emph{Statistics and
Computing}, 27(3), 659-678.

Grolemund, G. and Wickham, H. (2011). Dates and Times Made Easy with
lubridate. \emph{Journal of Statistical Software}, 40(3), 1-25. URL:
\url{http://www.jstatsoft.org/v40/i03/}.

Guntermann, K. L., Liu, C., \& Nowak, A. D. (2016). Price Indexes for
Short Horizons, Thin Markets or Smaller Cities. \emph{Journal of Real
Estate Research}, 38(1), 93-127.

Hastie, T., Tibshirani, R. and Friedman, J. (2008). \emph{The Elements
of Statistical Learning: Data Mining, Inference and Predcition.}
Springer.

Haurin, D. R., \& Hendershott, P. H. (1991). House price indexes: issues
and results. \emph{Real Estate Economics}, 19(3), 259-269.

Henry, L. and Wickham, H. (2019). purrr: Functional Programming Tools. R
package version 0.3.2. \url{https://CRAN.R-project.org/package=purrr}

Hill, R. (2013) Hedonic Price Indexes for Residential Housing: A Survey,
Evaluation and Taxonomy. \emph{Journal of Economic Surveys} 27(5),
879-914. \url{https://doi.org/10.1111/j.1467-6419.2012.00731.x}

Hill, R. C., Knight, J. R., \& Sirmans, C. F. (1997). Estimating capital
asset price indexes. \emph{Review of Economics and Statistics}, 79(2),
226-233.

Hill, R. and Trojanek, R. (2022) An evaluation of competing methods for
constructing house price indexes: The case of Warsaw. \emph{Land Use
Policy}, 120, 106226.

Hoesli, M., Giacotto, C., and Favarger, P. (1997) Three new real estate
price indices for Geneva, Switzerland. \emph{The Journal of Real Estate
Finance and Economics}, 15(1), 93-109.

Hyndman, R., Akram, M. and Archibald, B. (2008) The admissible parameter
space for exponential smoothing models. \emph{Annals of the Institute of
Statistical Mathematics}, 60(2), 407-426.

Hyndman R, Athanasopoulos G, Bergmeir C, Caceres G, Chhay L, O'Hara-Wild
M, Petropoulos F, Razbash S, Wang E, Yasmeen F (2019). \emph{forecast:
Forecasting functions for time series and linear models}. R package
version 8.7, \textless URL:
\url{http://pkg.robjhyndman.com/forecast}\textgreater.

Kuhn, M. (2019) caret: Classification and Regression Training. R
Package. \url{https://CRAN.R-project.org/package=caret}

McMillen, D. (2012). Repeat sales as a matching estimator. \emph{Real
Estate Economics}, 40(4), 745-773.

Maechler, M., Rousseeuw, P. Croux, C., Todorov, V., Ruckstuhl, A.,
Salibian-Barrera, M., Verbeke, T., Koller, M., Conceicao, E., and di
Palma, M.A.~(2019). robustbase: Basic Robust Statistics R package
version 0.93-5. \url{http://CRAN.R-project.org/package=robustbase}

Maguire, P., Miller, R., Moser, P. and Maguire, R. (2016) A robust house
price index using sparse and frugal data, \emph{Journal of Property
Research}, 33:4, 293-308,
\url{https://doi.org/10.1080/09599916.2016.1258718}

Mayer, M., Bourassa, S., Hoesli, M., and Scognamiglio, D. (2019).
Estimation and updating methods for hedonic valuation. \emph{Journal of
European Real Estate Research.} 12(1), 134-150.
\url{https://doi.org/10.1108/JERER-08-2018-0035}.

Meese, R. A., \& Wallace, N. (1997). The construction of residential
housing price indices: a comparison of repeat-sales, hedonic-regression,
and hybrid approaches. \emph{The Journal of Real Estate Finance and
Economics}, 14(1), 51-73.

Molnar, C. (2019) \emph{Interpretable Machine Learning: A Guide for
Making Black Box Model Explainable}. Leanpub. ISBN 978-0-244-76852-2.
\url{https://christophm.github.io/interpretable-ml-book/}

Moritz S, Bartz-Beielstein T (2017). ``imputeTS: Time Series Missing
Value Imputation in R.'' \emph{The R Journal}, \emph{9}(1), 207-218.
\url{doi:10.32614/RJ-2017-009} (URL:
\url{https://doi.org/10.32614/RJ-2017-009}).

Munneke, H. J., \& Slade, B. A. (2000). An empirical study of
sample-selection bias in indices of commercial real estate. \emph{The
Journal of Real Estate Finance and Economics}, 21(1), 45-64.

Nagaraja, C., Brown, L., \& Wachter, S. (2014). Repeat sales house price
index methodology. \emph{Journal of Real Estate Literature}, 22(1),
23-46.

Quigley, J. M. (1995). A simple hybrid model for estimating real estate
price indexes. \emph{Journal of Housing Economics}, 4(1), 1-12.

R Core Team (2019). R: A language and environment for statistical
computing. R Foundation for Statistical Computing, Vienna, Austria. URL
\url{https://www.R-project.org/}.

Ribiero, M., Singh, S. and Guestrin, C. (2016) Model-agnostic
Interpretability of Machine Learning. \emph{arXiv::1606.05386}.
\url{https://arxiv.org/abs/1606.05386}

Ribiero, M., Singh, S., and Guestrin, C. (2016b) ``Why Should I Trust
You?'': Explaining the Predictions of Any Classifier. \emph{Proceedings
of the 22nd ACM SIGKDD International Conference on Knowledge Discovery
and Data Mining.} pp 1135--1144, doi: 10.1145/2939672.2939778.

Rudin, C. and Carlson, D. (2019) The Secrets of Machine Learning: Ten
Things You Wish You Had Known Earlier to Be More Effective at Data
Analysis. \emph{Tutorials in Operations Research}
\url{https://arxiv.org/pdf/1906.01998v1}

Slack, D., Friedler, S., Scheidegger, C. and Roy, C.D. (2019) Assessing
the Local Interpretability of Machine Learning Models.
\url{https://arxiv.org/pdf/1902.03501.pdf}

Steele, M., \& Goy, R. (1997). Short holds, the distributions of first
and second sales, and bias in the repeat-sales price index. \emph{The
Journal of Real Estate Finance and Economics}, 14(1), 133-154.

Tofallis, C (2015). A better measure of relative prediction accuracy for
model selection and model estimation. \emph{Journal of the Operational
Research Society}, 66, 1352-1362. \url{doi:10.1057/jors.2014.103}

Wickham, H. (2016) \emph{ggplot2: Elegant Graphics for Data Analysis.}
Springer-Verlag New York.

Wickham, H., Francois, R., Henry, L. and Muller, K. (2019). dplyr: A
Grammar of Data Manipulation.
\url{https://CRAN.R-project.org/package=dplyr}

Wickham, H. and Henry, L. (2019). tidyr: Easily Tidy Data with
`spread()' and `gather()' Functions. R package version 0.8.3.
\url{https://CRAN.R-project.org/package=tidyr}

Wright, M. and Ziegler, A. (2017) ranger: A Fast Implmentation of Random
Forests for High Dimensional Data in C++ and R. \emph{Journal of
Statistical Software}, 77(1), 1-17.

Van de Minne, Alex et al.~``Using Revisions as a Measure of Price Index
Quality in Repeat-Sales Models.'' Journal of Real Estate Finance and
Economics 60 (May 2020): 514--553

Xie, Y. (2019). knitr: A General-Purpose Package for Dynamic Report
Generation in R. R package version 1.23.

Xu, X. and Zhang, Y. (2022) Second-hand house price index forecasting
with neural networks, \emph{Journal of Property Research}, 39:3,
215-236.

Zeileis, A. and Grothendieck, G. (2005). zoo: S3 Infrastructure for
Regular and Irregular Time Series. \emph{Journal of Statistical
Software}, 14(6), 1-27. \url{doi:10.18637/jss.v014.i06}

\end{document}
